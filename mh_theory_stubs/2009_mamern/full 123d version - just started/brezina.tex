\documentclass{elsart}

\usepackage{epsfig}
\usepackage[utf8]{inputenc}

%\usepackage{array}

% ***************************************** PACKAGES
%\usepackage[active]{srcltx}
\usepackage{amsmath}
\usepackage{amsfonts}
\usepackage{amssymb}
%\usepackage{amsthm}
%\usepackage{fancyhdr}
%\usepackage{mathbbol}
%\usepackage{bbm}
% ***************************************** THEOREMS
%\newtheorem{thm}{Theorem}[section]
%\newtheorem{proposition}[thm]{Proposition}
\newtheorem{definition}[thm]{Definition}
%\newtheorem{remark}[thm]{Remark}
%\newtheorem{lemma}[thm]{Lemma}
%\newtheorem{corollary}[thm]{Corollary}
%\numberwithin{equation}{section}
% ***************************************** SYMBOLS
\def\div{{\rm div}}
\def\Lapl{\Delta}
\def\grad{\nabla}
\def\supp{{\rm supp}}
\def\dist{{\rm dist}}
%\def\chset{\mathbbm{1}}
\def\chset{1}

\def\Tr{{\rm Tr}}
\def\sgn{{\rm sgn}}
\def\to{\rightarrow}
\def\weakto{\rightharpoonup}
\def\imbed{\hookrightarrow}
\def\cimbed{\subset\subset}
\def\range{{\mathcal R}}
\def\leprox{\lesssim}
\def\argdot{{\hspace{0.18em}\cdot\hspace{0.18em}}}
\def\Distr{{\mathcal D}}
\def\calK{{\mathcal K}}
\def\FromTo{|\rightarrow}
\def\convol{\star}
\def\impl{\Rightarrow}
%\DeclareMathOperator*{\esslim}{esslim}
%\DeclareMathOperator*{\esssup}{ess\,sup}
%\DeclareMathOperator{\ess}{ess}
%\DeclareMathOperator{\osc}{osc}
%\DeclareMathOperator{\curl}{curl}

%\def\Ess{{\rm ess}}
%\def\Exp{{\rm exp}}
%\def\Implies{\Longrightarrow}
%\def\Equiv{\Longleftrightarrow}
% ****************************************** GENERAL MATH NOTATION
\def\Real{{\rm\bf R}}
\def\Rd{{{\rm\bf R}^{\rm 3}}}
\def\RN{{{\rm\bf R}^N}}
\def\D{{\mathbb D}}
\def\Nnum{{\mathbb N}}
\def\Measures{{\mathcal M}}
\def\d{\,{\rm d}}               % differential
\def\sdodt{\genfrac{}{}{}{1}{\rm d}{{\rm d}t}}
\def\dodt{\genfrac{}{}{}{}{\rm d}{{\rm d}t}}

\def\vc#1{\mathbf{\boldsymbol{#1}}}     % vector
\def\tn#1{{\mathbb{#1}}}    % tensor
\def\abs#1{\lvert#1\rvert}
\def\Abs#1{\bigl\lvert#1\bigr\rvert}
\def\bigabs#1{\bigl\lvert#1\bigr\rvert}
\def\Bigabs#1{\Big\lvert#1\Big\rvert}
\def\ABS#1{\left\lvert#1\right\rvert}
\def\norm#1{\bigl\Vert#1\bigr\Vert} %norm
\def\close#1{\overline{#1}}
\def\inter#1{#1^\circ}
\def\ol#1{\overline{#1}}
\def\ul#1{\underline{#1}}
\def\eqdef{\mathrel{\mathop:}=}     % defining equivalence
\def\where{\,|\,}                    % "where" separator in set's defs
\def\timeD#1{\dot{\overline{{#1}}}}

% ******************************************* USEFULL MACROS
\def\RomanEnum{\renewcommand{\labelenumi}{\rm (\roman{enumi})}}   % enumerate by roman numbers
\def\rf#1{(\ref{#1})}                                             % ref. shortcut
\def\prtl{\partial}                                        % partial deriv.
\def\Names#1{{\scshape #1}}
\def\rem#1{{\parskip=0cm\par!! {\sl\small #1} !!}}

\def\Xint#1{\mathchoice
{\XXint\displaystyle\textstyle{#1}}%
{\XXint\textstyle\scriptstyle{#1}}%
{\XXint\scriptstyle\scriptscriptstyle{#1}}%
{\XXint\scriptscriptstyle\scriptscriptstyle{#1}}%
\!\int}
\def\XXint#1#2#3{{\setbox0=\hbox{$#1{#2#3}{\int}$}
\vcenter{\hbox{$#2#3$}}\kern-.5\wd0}}
\def\ddashint{\Xint=}
\def\dashint{\Xint-}

% ******************************************* DOCUMENT NOTATIONS
% document specific
\def\rh{\varrho}

\def\th{\vartheta}
\def\vl{{\vc{u}}}
\def\vx{\vc{x}}
\def\vX{\vc{X}}
\def\vr{\vc{r}}
\def\veta{\vc{\eta}}
\def\dx{\,\d\vx}
\def\dt{\,\d t}
\def\bulk{\zeta}
\def\cS{\close{S}}
\def\eps{\varepsilon}
\def\phi{\varphi}
\def\Bog{{\mathcal B}}
\def\Riesz{{\mathcal R}}
\def\distr{\mathcal D}
\def\Item{$\bullet$}
\def\iset#1{{\mathcal{#1}}}
\def\Hdiv{{\bf H^{\rm div}}}

\def\MEtst{\mathcal T}
%***************************************************************************



\begin{document}

\begin{frontmatter}
 \title{Convergence analysis
 for non-compatible fracture flow discretizations}


\author[NTI]{Jan B\v rezina}
\ead{jan.brezina@tul.cz}

\address[NTI]{New Technologies Institute, 
Technical University in Liberec, 
Studentsk\'a 2/1402,
461 17  Liberec, 
Czech Republic
}

\begin{abstract}
Simulation of the underground water flow in real 3D domains requires also modeling of the flow in
2D fractures and their 1D intersections. For each dimension we consider  
the continuity equation and Darcy's law as a model of the stationary saturated flow.
The water flux between individual dimensions is assumed to be proportional to the pressure difference. The classical discretization schemes requires the alignment of computational meshes between dimensions. We present a mixed-hybrid formulation of the problem and two approximations of the communication terms that relax the alignment condition.
We perform convergence analysis of these approximations.
\end{abstract}

\begin{keyword}
Numerical analysis\sep Mixed formulation\sep Fracture flow 

MAC: 65N12, 35M32
\end{keyword}

\end{frontmatter}




%\title{Convergence analysis for \\a non-compatible discretization\\
% of multidimensional fracture flow problem}






%********************************************************
\section{Introduction}
We consider a porous media flow in a 3D domain which contains a network of 2D polygonal fractures and their 1D intersections -- chanels. The 3D volume $\tilde\Omega_3$ is splitted by the fracture planes into subdomains $\Omega_3^i$, $i\in I_3$; the set of open 2D polygons $\tilde\Omega_2$ is broken into open convex subpolygons $\Omega_2^i$, $i\in I_2$ by the lines of their intersections; the set $\tilde\Omega_1$ of these 1D intersections is docomposed into line segments $\Omega_1^i$, $i\in I_1$ by their cross points $\Omega_0^i$, $i\in I_0$.

In particular, we have 
\begin{equation}
  \Omega_{d-1} \subset \prtl\Omega_d \setminus \prtl\tilde\Omega_d \quad \text{for} d=1,2,3;
\end{equation}
where $\Omega_d$ is union of $\Omega_d^i$, $i\in I_d$.

On every subdomain $\Omega_d^i$, we consider linear flow equations
\begin{equation}
\label{flow_eq} 
   \div \vl= f;\quad  \vl=-\tn K \grad p,
\end{equation}
where the velocity $\vl$ and the pressure $p$ are unknowns, while given data are the volume density $f$ of the water sources,
and the hydraulic permeability tensor $\tn K$.
The velocity $\vl$ and the permeability tensor $\tn K$ live on the tangent space of the particular subdomain. 

Water exchange between subdomains is given by pressure gradient. Whenever $\Omega_d^i$ is intersection of manifolds
$\Omega_{d+1}^j$, $j\in I_i$, we prescribe Newton-like condition
\[
    \vl_j(\vc x)\cdot\vc n_j(\vc x) = \sigma_j(\vc x)(p_j(\vc x)-p_i(\vc x),\quad  \forall\, j\in I_i,\ \vc x\in\Omega_d^i
\]
where $\vl_j(\vc x)$ and $p_j(\vc x)$ are traces of the velocity and the pressure on the boundary of manifold $\Omega_{d+1}^i$, 
$p_i(\vc x)$ is pressure on $\Omega_d^i$, and $\sigma_j(\vc x)$ is the water transfer coefficient of the manifold $\Omega_{d+1}^j$. The additional source $\tilde f_d(\vc x)$ on the manifold $\Omega_d^i$, $d=1,2$ has a form
\[
  \tilde f_d(x)=\sum_{j\in I_i} \vl_j(x)\cdot\vc n_j(x).
\]
On the exterior boundary $\Gamma_d=\prtl\Omega \cap \prtl\Omega_d$ of the domain $\Omega_d$, we prescribe general Newton boundary condition
\[
   \vl(\vc x)\cdot\vc n(\vc x) = \alpha_d(\vc x)(p(\vc x)-P_d(\vc x)),
\]
where $\alpha_d$ and $P_d$ are given data.

In order to obtain good approximation of the velocity field for the transport model, we use discretization
based on mixed-hybrid formulation. We denote $\tilde\Omega_d^i$, $i\in \tilde I_d$ a further decomposition of
$\Omega_d^i$, $i\in I_d$ and we denote 
\[
\Gamma_d^i=\close{\Omega_d^i}\setminus\bigcup_{i\in\tilde I_d} \tilde\Omega_d^i.
\]

the interior and boundary edges of the decomposed set $\Omega_d^i$. Then we introduce spaces
\begin{gather}
    \label{Vspace}
        V=V_3\times V_2\times V_1=
                \prod_{d\in{3,2,1}} \prod_{j\in \tilde I_d} H(\div,\tilde\Omega_d^j)\\
    \label{Pspace}  
        P=P_3\times P_2\times P_1\times \mathring{P}_3\times \mathring{P}_2 \times \mathring{P}_1,\\
    \notag
        P_d=L^2(\Omega_d),\quad \mathring{P}_d=\prod_{i\in I_d}H^{1/2}(\Gamma_d^i).
\end{gather}
In following, we use notation $\vl_d^j$ for components of $\vl\in V$ and $p_d$, $\mathring{p}_d^i$ for components of $p\in P$. Noting, that on $\Omega_d^i$, $d\in 1,2$, we have one pressure $p_d$ on the fracture $p_d$ and possibly several traces $\mathring{p}_d^i$ of the pressure on neighbouring subdomains. 

For every $j\in\tilde I_d$ we denote 
\[
   i(j) \in I_d\ :\ \Omega_d^j \subset \Omega_d^{i(j)} 
\]
the index of "mother" domain of $\Omega_d^j$. Now we define mixed-hybrid solution as follows
\begin{definition}
We say that pair $(\vl,p)\in V\times P$ is MH-solution of the problem if it satisfy 
abstract saddle point problem
\begin{align}
        \label{Saddle1}
 a(\vl,\vc w) + b(\vc w, p) &= 0 &&\forall \vc w\in V,\\
        \label{Saddle2}
 b(\vl, q) - c(p,q) &= \langle F, q \rangle &&\forall q \in P,
\end{align}
where
\begin{align*}
 a(\vc v,\vc w)&=\sum_{d=1,2,3}\sum_{j\in \tilde I_d} \int_{\tilde\Omega_d^j} \vc v_d^j \tn K_d^{-1} \vc w_d^j,\\
 b(\vc v,q)&=\sum_{d=1,2,3}\sum_{j\in \tilde I_d} 
        \left(
        \int_{\tilde\Omega_d^j} -\div\vc v_d^j q_d
        +\int_{\prtl\tilde\Omega_d^j}
                 (\vc v_d^j \cdot \vc n)\mathring{q}_d^{i(j)}
        \right),\\
 c(p,q)&=\sum_{d=1,2,3} \sum_{j\in \tilde I_{d-1}}
          \left(
          \int_{\tilde\Omega_{d-1}^j} (p_{d-1}-\mathring{p}_{d}^j)(q_{d-1}-\mathring{q}_{d}^j)
          +\int_{\Gamma_d} \alpha_d \mathring{p}_d \mathring{q}_d
          \right),\\
 \langle F, q\rangle &= -\sum_{d=1,2,3}\left(\int_{\Omega_d} f_d p_d +
           \int_{\Gamma_d}  \alpha_d P_d \mathring{q}_d\right).
\end{align*}
\end{definition}
To simplify formula for $c(p,q)$, we have formally set $\tilde I_{0} = \emptyset$ since the first integral is taken 
only over 1D and 2D communication interfaces, while the second term corresponds to the Newton boundary condition for 
every dimension. In the term $c(p,q)$, the symbols $\mathring{p}$ and $\mathring{q}$ should be understood as 
elemnets of $L^2 \supset H^{1/2}$.

In the following we consider lowest order approximation of the MH-formulation. Assume that $\Omega_d$ are polyhedrons 
triangulated by simplexes $\tilde\Omega_d^j$, $j\in \tilde I_d$. 
Then, we approximate the space $H(\div,\tilde\Omega_d^j)$ by the Raviart-Thomas space $RT_0(\tilde\Omega_d^j)$ (see \cite{fortin_mixed_1991}) and the spaces $L^2(\Omega_d)$ and $H^{1/2}(\Gamma_d^i)$ by piecewise constant functions on elements and their edges respectively. (for details see \cite{maryska_mixed-hybrid_1995}).





We consider linear model for a flow in saturated porous media
\begin{equation}\label{Darcy}
        \vl_d=-\tn K_d \grad p_d \quad \text{on }\tilde \Omega_d
\end{equation}
and the continuity equation
\begin{equation}\label{RC}
        \div \vl_d = F_d        \quad \text{on }\tilde \Omega_d,
\end{equation}


For the sake of clarity we present ideas on the simple 2D-1D case. Let us consider a 2D domain $\Omega_2\subset \Real^2$
splitted into two subdomains by a 1D fracture $\Omega_1\subset\Omega_2$.
We denote $ \tilde\Omega_2=\Omega_2\setminus\Omega_1$ and $\tilde\Omega_1=\Omega_1$.
To avoid technical difficulties we assume that $\Omega_2$ have polygonal boundary and $\Omega_1$ is a straight line. The flow on the domain $\Omega_d$ ($d=1,2$) is described by the velocity $\vl_d$ and the pressure
$p_d$. These state variables has to satisfy Darcy's law
where $\tn K_d$ is (tensor of) the hydraulic conductivity, $F_2=f_2$ and $F_1=f_1+q$
are water sources, while $q$ denotes the outflow from 2D domain. We consider a non-separating crack, which means that the pressure is continuous across the crack and the sum of outflow from the walls of the fracture is equal to the fracture inflow,
namely  
\begin{gather*}
 	p_2^{+} = p_2^{-}\quad \text{on }\Omega_1,\\
 	[\vl_2]_{\Omega_1}:= (\vc \vl_2^{+}\cdot\vc n^{+} + \vc u_2^{-}\cdot\vc n^{-}) = q.
\end{gather*}
Since the pressure is continuous we can prescribe
\[
 q=\sigma(p_2|_{\Omega_1} - p_1),
\]
where $\sigma$ is an interchange coefficient, we take $\sigma=1$. The system is completed by the boundary conditions
\begin{align*}
%        \label{ClassDirichBC} 
        p_d = p^D&\quad \text{on }\Gamma_d^D,\\
%        \label{ClassNeumanBC}
        \vl_d\cdot\vc n= u^N&\quad \text{on }\Gamma_d^N.
\end{align*}
where $\Gamma_d^D$ is Dirichlet and $\Gamma_d^N$ Neumann part of the boundary $\prtl\Omega_d$.

\section{Mixed-hybrid formulation on aligned meshes}
Let $\{\Omega_d^i\}$, $i\in I_d$ be a decomposition of domain $\Omega_d$ into disjoint subdomains
that satisfy the alignment condition
\begin{equation}\label{compatible}
 	\Omega_1\subset \Gamma_2,\quad \Gamma_d:=\bigcup_{i\in I_d} \prtl\Omega_d^i\setminus\prtl\Omega_d,
\end{equation}
where $\Gamma_d$ is union of interior faces. We multiply the equations \eqref{Darcy},
\eqref{RC} by suitable test functions and integrate over the individual subdomains.
We integrate by parts in \eqref{Darcy} and we treat traces of the pressure as
an independent variable (for details see~\cite{Maryska}). Finally we obtain mixed-hybrid formulation of the problem specified in the previous section.

\begin{thm}{Definition 1.}
We say that $(\vl,p)$ is MH-solution of the problem if the composed velocity $\vl$ and
the composed pressure $p$ belong to the spaces
\begin{gather}
	\label{Vspace}
	\vl=(\vl_2,\vl_1)\in V=V_2\times V_1=
		\prod_{i\in I_2} H(\div,\Omega_2^i)\times \prod_{i\in I_1} H(\div,\Omega_1^i)\\
	\label{Pspace}	
    p=(p_2,p_1,\ol{p}_2,\ol{p}_1)\in P=P_2\times P_1\times \ol{P}_2\times\ol{P}_1,\\
    \notag
    P_d=L^2(\Omega_d),\quad \ol{P}_d=H^{1/2}(\Gamma_d\cup\Gamma_d^N).
\end{gather}
and they satisfy abstract saddle point problem
\begin{align}
        \label{Saddle1}
 a(\vl,\vc w) + b(\vc w, p) &= \langle G(p^D), \vc w \rangle &&\forall \vc w=(\vc\phi_2,\vc\phi_1)\in V,\\
        \label{Saddle2}
 b(\vl, q) - c(p,q) &= \langle F(f,u^N), q \rangle &&\forall q=(q_2,q_1,\ol{q}_2,\ol{q}_1) \in P,
\end{align}
where
\begin{gather*}
 a(\vc v,\vc w)=\sum_{d=1,2}\sum_{i\in I_d} \int_{\Omega_d^i} \vc v_d \tn K_d^{-1} \vc w_d\\
 b(\vc v,q)=\sum_{d=1,2}\sum_{i\in I_d} \int_{\Omega_d^i} -\div\vc v_d q_d
 	+\int_{\Gamma_d} [\vc v_d]\ol{q}_d+\int_{\Gamma_d^N} (\vc v_d\cdot \vc n)\ol{q}_d\\
 c(p,q)=\int_{\Omega_1} (p_1-\ol{p}_2)(q_1-\ol{q}_2).
 \label{Cform}
\end{gather*}
\end{thm}
Note that in definition of bilinear form $c(p,q)$, we need trace $\ol{p}_2$ on $\Omega_1$,
which is available because of the condition \eqref{compatible}.

The existence and uniqueness of the MH-solution is a direct corollary of Theorem 1.2 in \cite{Brezzi}.
Furthermore, one can use Raviart-Thomas elements to construct approximation spaces and use again theory
from \cite{Brezzi} to get an $O(h)$ estimate:
\begin{equation}\label{CompatibleConvergence}
 \norm{\vl-\vl_h}_{V} + \norm{p-p_h}_{P} \le C h\big(\norm{\vl}_{H^1} + \norm{p}_{H^1}\big).
\end{equation}
Note that according to the regularity theory for the linear elliptic equations one can expect $p\in W^{2,p}$ and $\vl\in W^{1,p}$ for all $1<p<\infty$, provided data from $L^\infty$ and certain beter regularity of the boundary conditions.

\section{Mixed-hybrid formulation on non-compatible meshes}
In practical applications, we frequently use statistically generated fractures. In this situation,
it could be nearly impossible to produce a regular mesh satisfying the alignment condition
\eqref{compatible}.  In order to relax this condition, we have to construct an approximation of
the trace of the pressure on the fracture $\Omega_1$. Let $\{\Omega_2^i\},\ i\in I_2$ be a regular triangular decomposition of $\Omega_2$ with diameter of elements bounded by $h$ and let
\[
        \Omega_1^i=\Omega_2^i \cap \Omega_1,
\] be elements of the induced decomposition of $\Omega_1$.
On these decompositions we consider spaces $V^h$ and $P^h$ similar to \eqref{Vspace} and \eqref{Pspace}. Splitting further the triangles intersecting $\Omega_1$, we obtain an aligned decomposition of $\Omega_2$ on which we can build spaces $V$ and $P$.
Finally, we denote $\Omega_{12}$ the union of $\Omega_2$-subdomains intersecting $\Omega_1$
\[
  \Omega_{12}=\bigcup_{i\in I_{1}} \Omega_2^i,\quad
   \text{where }  I_{1}=\{i\in I_2| \Omega_1^i  \ne \emptyset \}.
\]

Let $\Pi$ be a continuous linear operator from $P_h$ to the functions piecewise continuous on
subdomains $\Omega_2^i$, $i\in I_1$. Then we can define an approximation $T$ of the trace operator for the functions $p\in P_h$ by the formula $T(p)=Tr_{\Omega_1}(\Pi(p))$, locally on each triangle $\Omega_2^i$, $i\in I_1$. In particular, we can use an average approximation
\[
   \Pi_0 |_{\Omega_2^i}(p) = \frac{1}{\abs{\Omega_2^i}}
   		\int_{\Omega_2^i} p_2 \dx,\quad \forall i\in I_{1}
\]
or a piecewise linear approximation $\Pi_1$ such that
\[
  \int_S \Pi_1|_{\Omega_2^i}(p) = \int_S \ol{p}_2
\]
for every side $S$ of the triangle $\Omega_2^i$, $i\in I_{1}$.

The operator $T$ can be used to extend the trace component $\ol{p}_2$ of the space $P_h$ on the fracture $\Omega_1$. Consequently, we  can treat the space $P_h$ as a subspace of $P$ with a norm
\begin{equation}\label{PspaceExtension}
 	\norm{f}_P=\norm{f}_{P_h}+\norm{T f}_{H^{1/2}(\Omega_1)}.
\end{equation}
In view of this convention, one can use Definition 1 with spaces $V_h \subset V$ and $P_h\subset P$ to introduce a semi-discrete solution $(\vl_h,p_h)$. Then again,
Theorem 1.2 in \cite{Brezzi} implies the existence and uniqueness of the solution.

Now we want to compare the MH-solution $(u,p)\in V\times P$ to the semi-discrete solution $(\vl_h,p_h)\in V_h\times P_h$. Following the proof of Proposition 2.11 in \cite{Brezzi}, we can show
\begin{equation}\label{MainEst}
 	\norm{\vl-\vl_h}_V + \norm{p-p_h}_P \le C\big(
 		\inf_{\vc v_h \in V_h}\norm{\vl - \vc v_h}_{V} + \inf_{q_h\in P_h}\norm{p-q_h}_{P}
                \big).
\end{equation}
Taking $\vc v_h=\vl-\vl^*$, where $\vl^*$ is a divergence free extension of $[\vl]_{\Omega_1}$
to $\Omega_{21}$, we can bound the first term by $O(h)$ times $L^2$-norm of $[\vl]_{\Omega_1}$.
Further, we take $q_h$ equal to the projection of $p$ on the space $P_h$, then according to \eqref{PspaceExtension} we get
\[
  \norm{p-q_h}_{P} = \norm{p-T p}_{H^{1/2}(\Omega_1)}\le C\Big(\sum_{i\in I_1} \norm{p - \Pi p}_{H^1(\Omega_2^i)}^2\Big) ^\frac12.
\]
If $\Pi$ preserves polynomials up to the order $k$, the standard approximation estimates
(see \cite{FEMHandbook} Theorem 16.2) leads to
\[
 \norm{p-\Pi p}_{1,\Omega_2^i}\le C\abs{\Omega_2^i}^{\frac12-\frac1q} h^k \abs{p}_{k,q,\Omega_2^i}.
\]
Since we assume regular triangulation, we have $\abs{\Omega_2^i}\le C h^N$ and $\abs{I_1} \le C h^{1-N}$,
where $N=2$ is dimension of $\Omega_2$.
Then we conclude
\[
        \norm{p-Tp}_{H^{1/2}(\Omega_1)}\le C h^{N(\frac12-\frac1q)+k}\Big(\sum_{i\in I_1} \abs{p}_{k+1,q,\Omega_2^i}^2 \Big)^\frac12\le
        C h^\alpha \abs{p}_{k+1,q,\Omega_2}
\]
where 
\begin{equation}\label{ApproxOrder}
        \alpha= N\Big(\frac12-\frac1q\Big)+k+(1-N)\frac{q-2}{2q}=k+\frac{1}{2}-\frac{1}{q}.
\end{equation}

\section{Conclusion}
We have proposed two approximations of the original MH-problem on the non-aligned meshes. 
The first is based
on the operator $\Pi_0$, which preserves only polynomials of zero order. Hence, $\alpha<1/2$ for all $q$ and we obtain a suboptimal convergence compared to \eqref{CompatibleConvergence}. In the later case, the approximation is based on the operator
$\Pi_1$, which preserves polynomials of the first order. We get $\alpha=1$ for $q=2$ and an optimal  convergence rate $O(h)$.
In both cases, we have to assume certain regularity of the exact solution. Although we did our analysis
only in 2D case with simple geometry, the abstract formulation and the results hold also for more complicated geometries and 3D-2D communication. 

{\bf Acknowledgement: This work was supported by GA\v CR 205/09/P567.}

\begin{thebibliography}{99}
\bibitem{Brezzi} F. Brezzi, M. Fortin,
{\it Mixed and Hybrid Finite Element Methods}.
Springer Verlag, Berlin/Heidelberg, 1991
\bibitem{FEMHandbook} P.G. Ciarlet, J.L. Lions,
{\it Handbook of Numerical Analysis : Finite element method (Part~1)}.
 North-Holland, Amsterdam, 1991
\bibitem{Maryska} J. Mary\v{s}ka,  M. Rozlo\v{z}n\'{\i}k, M. T\r{u}ma.
{\it Mixed-hybrid finite element approximation  of the potential fluid flow problem}.
J. Comput. Appl. Math., Vol. 63, (1995), pp. 383--392.

\end{thebibliography}

\end{document}









